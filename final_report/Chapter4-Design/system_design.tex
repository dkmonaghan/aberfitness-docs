\subsection{Persistence}
    \par
    Object relational mapping \textit{ORM} provides a way for developers to write database agnostic models and queries for persisting data. 
    
    \par
    To keep code both testable and maintainable, services written in .NET Core and Java use an ORM layer. This generates database schemas based on the application's model classes

    \par
    \textit{MariaDB}\cite{MariaDB} was selected for several reasons: Firstly, the database provides prescience features required by the system. Secondly, native .NET core and Java connectors exist as framework packages. Finally, the software is GPL licensed avoiding future issues with proprietary software.

\subsection{Model View Controller}
    \par
    All implementations use the Model View Controller \textit{(MVC)}
    pattern. This enhances testability by keeping constraints and functional logic within a model class. Controllers use an instance of the model and prepare data for return in a view.

    \par
    Dependency injection and reflection were used to test .NET core and Java components. These design patterns allow developers to replace concrete types with mocks at runtime based on the class interface.

\subsection{Packaging}
    \par
    We choose to use docker images to deploy our images, these can be combined into an application stack defined in a docker compose file. This also allows us to define internal networks, start-up order and filesystem volumes.

    \par
    The deployed stack can be restarted or updated with the `\textit{docker-compose}' command, whilst individual containers can be managed using the docker commands directly.