\chapter{Design}

\section{System Overview}
The \textit{Aber Fitness} system is broken down into a number of microservices in order to aid portability, scalability and promotes a more maintainable codebase. After reviewing the initial project specification, the system was broken down into the following microservices:

\begin{itemize}

	\item \textbf{Booking \& Facilities} - The \textit{Aber Fitness} system must offer the functionality for users to be able to schedule bookings at sports venues such as swimming pools and squash courts. This microservice is also called upon by the "Ladders" system to create bookings for competitions.

	\item \textbf{Challenges} - One of the requirements of the system is that it offers the ability for users to be given activity challenges, such as completing a number of steps in a specific timeframe. These challenges can also be 'group' challenges where a number of users can compete against one another to achieve goals such as furthest distance walked in a week, etc.

	\item \textbf{Communications} - This microservice provides an API for communicating with users via email. It does not present any form of web UI, and users do not directly interact with it. It simply acts as an abstraction layer between other microservices and sending out emails. With this design, the system could also be expanded to send out text messages, push alerts, etc.

	\item \textbf{Fitbit Ingest Service} - At launch, the \textit{Aber Fitness} platform is required to allow users to connect their \textit{Fitbit} accounts to the system in order to import their activity data. The concept of an "ingest service" was created in order to allow future platform expansion to support services such as Apple's \textit{HealthKit} and other fitness tech providers. This architectural design means that activity data can be normalised by a number of "ingest services" before being passed through to the \textit{Health Data Repository} service for storage. 

	\item \textbf{Gatekeeper} - \textit{Gatekeeper} is \textit{Aber Fitness}'s OpenID Provider, and handles all authentication within the system. User passwords and metadata is stored within \textit{Gatekeeper}, and \textit{Gatekeeper}r is responsible for providing a single sign-on service for all of the various microservices. Microservices will also utilise \textit{Gatekeeper} to obtain tokens to utilize eachother's APIs.

	\item \textbf{GLaDOS} - \textit{GLaDOS} is the centralised logging and auditing system for \textit{Aber Fitness}. It presents a REST API which other microservices can use to store audit data, such as when a user's data was accessed, modified, or deleted. \textit{GLaDOS} is also responsible for providing a Status page which displays the availability of all the other microservices.

	\item \textbf{Health Dashboard} - \textit{Health Dashboard} is the first interface users will encounter when logging in to \textit{Aber Fitness}, it provides the user with an overview of their recent activity as well as providing updates on any challenges or ladder competitions the user may be involved in.

	\item \textbf{Health Data Repository} - The \textit{Health Data Repository} service is responsible for providing an API to ingest and export activity data. It can accept normalised activity data from the Ingest Services, and provides API endpoints in order to provide user activity data to other microservices. 

	\item \textbf{Ladders} - \textit{Ladders} is responsible for organising and managing ladder style competitions among users of the system, where users can compete in sporting championships for a variety of competitive sports such as tennis, etc. The \textit{Ladders} system also makes use of the \textit{Booking Facilities} microservice in order to automatically book venues for upcoming competitive events.

	\item \textbf{User Groups} - Within the \textit{Aber Fitness} system, the \textit{Challenges} can also be made competitive amongst users of a group. For example, a group may consist of a few friends or an entire office department. The users within these groups can then compete to see who can achieve the most steps in a single day, for example. The \textit{User Groups} service is responsible for managing users into groups, and allowing users to leave and join other groups.

\end{itemize}