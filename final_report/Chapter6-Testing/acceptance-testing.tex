\section{Acceptance Testing}
\par
During our first meeting, the project's functional requirements were broken down into a number of spreadsheets, each based on the service which would implement them. The client was then contacted to clarify any requirements which were ambiguous to the group. Acceptance tests for each micro-service were added to a spreadsheet based on these original and revised requirements.

\par
Acceptance tests are completed manually by completing the steps defined by the document. Each test is marked as pass or fail based on the criteria also set out in the document, allowing the individual microservices, and by extension the entire project, to be assessed on its fitness for purpose.

\par
A fresh deployment of the system was performed shortly after a code freeze and the full suite of acceptance tests were completed. This round of testing highlighted 23 instances where the implementation deviated from the client's requirements.

\par
Bugs were assessed on a case-by-case basis, taking account of the impact. Fixes which had a low potential for regression and addressed high impact problems were implemented and merged after the code freeze. This process resolved 11 of those implementation issues, resulting in almost half of the failed acceptance tests becoming passes in the space of a few hours, and with no additional issues appearing as side-effects. This shows that internal acceptance testing was a highly valuable process that resulted in a noticable improvement in the final product.
