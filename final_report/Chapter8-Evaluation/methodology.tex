\section{Development Methodology}

  \subsection{Evaluation}
  \par
  The group met for weekly meetings at the start of the project to define a top-level system design on a whiteboard. These sessions were initially productive, as work was distributed to subteams and a unified approach for the various microservices was decided. As the project went on, the usefulness of these meetings was questioned. Attempts to design by committee would rapidly devolve into either a few interested parties talking about a single service, or every developer inputting an opinion on simple decisions.

  \par
  A vote unanimously agreed to change the group meeting into a session where every developer would work on implementation in the same room instead. This refocused teams onto their own design, whilst allowing an individual to still propose system-level questions or ideas in person. Productivity also rapidly increased as knowledge of common problems and solutions were shared amongst the group.

  \par
  Throughout the project there have been instances where the requirements were not clear. Reasons ranged from multiple interpretations, noun and verb conflation, and insufficient detail. Slack was used as a communication mechanism with the client, which allowed the client to choose an appropriate time to respond to requests, which could be sent at any time, while also enabling real time discussion. These communications taking place over recorded medium meant that an answer could be easily converted into an additional requirement later.

  \par
  As the project progressed, Scrumban-related processes were skipped increasingly. Project boards weren't being updated or tracked and work was completed without an associated issue or project board entry. As there were eight members of the group, it became increasingly difficult to track the work that individual teams had remaining. Ultimately this lead to deadlines being missed, as teams had to estimate their work left to complete rather than having a metric to rely on.

  \par
  As the scope of each unit's work was not defined, branches would often contain multiple features and fixes. Pull requests would often alternate between being too large to sensibly review, or a single fix spread across several pull requests. Additionally, developers would typically review within a subteam, instead of examining other services' code. This lead to code reviews being approved without the approver properly reviewing the change. This culminated into several simple bugs and version control issues, which would likely have been spotted, such as incorrect equality comparisons being merged into the codebase.

  \subsection{Future Improvements}
  Many of the aforementioned issues would have been resolved by nominating a designated team leader. This person would ensure the work being completed has an accompanying issue which exists on a project board. They would also ensure that reviews are completed correctly and that different subteams are reviewing each other's code.

  \par
  Additional work meetings in the week would reduce the timespan between a difficult problem being discovered and co-developers being able to help in diagnosis and resolution. It would also create another opportunity for knowledge transfer across subteams in person.
