During the analysis and planning stages it quickly became apparent that a central authentication and authorization system would be required.  The Gatekeeper service is responsible for providing this, and does so by acting as both an OAuth 2.0 Authorization Server, and an OpenID Connect Identity Provider (IDP).  Such a setup provides a number of benefits, including seamless Single Sign-On (SSO), relative ease of implementation due to the abundance of available libraries, separation of concerns by decoupling authentication \& authorization from each microservice, and the theoretical ability to allow third-party access to AberFitness APIs in the future.

\subsubsection{IdentityServer 4}

Rather than attempting to implement our own OAuth 2.0 server, it seemed beneficial in terms of both security and development time to use an existing solution.  After thorough investigation of the available implementations, IdentityServer 4 was chosen as it not only fully implements the specification of OAuth 2.0 and OpenID Connect, but is also certified by the OpenID Foundation, and is part of the .NET Foundation.  Thorough documentation is available, the project is open-source with a permissive license, and the library is specifically designed for .NET Core 2.

\subsubsection{OAuth Overview}

Two basic OAuth 2.0 flows are used within AberFitness - Authorization Code (including the hybryid flow OpenID Connect extension) for applications with a GUI wishing to identify users, and Client Credentials for API authorization between the various microservices.




There was some consideration as to whether client credentials was the correct flow to use for authorization when obtaining user resources within AberFitness.  Typically you would require an access token specific to that user, obtained using the authorization code flow, in order to access resources such as activity data belonging to that user.  Eventually it was decided that the authorization code flow would be more relevant to third-party applications wishing to access our APIs.  As all microservices within AberFitness can be considered as belonging to the same overall system, it seemed that the client credentials flow was sufficient to authorize access to data from other services within the AberFitness system.

\subsubsection{Implementation Details}

data protection keys

no editing

schema complexity

\subsubsection{Alternatives}

.NET alternatives

Java implementations

SAML

API gateway
