\subsubsection{MVC Architecture}
\par
The core architecture followed the MVC style recommended by .NET Core. The initial design was completed by creating models and defining how they would interact. Listed below are the core models which outlined the core layout.

\begin{itemize}

	\item The first model was \textit{profile}, this was used to store information regarding the user's availability and their preferred location. The model was also used to link the user to their ladder and their position within it, when their request to join one was accepted.

	\item The second model was the \textit{ladder}, this was used as a repository to store a collection of users. The ladder name is the only field which is exposed to the end user. Internally, it holds a reference to all the position objects for the current ladder. It also holds references to any users who have sent a request to join a ladder, but are pending approval.

	\item The third model is the \textit{position}. This is used to link a user profile to a ladder, it is not visible to the user but is a critical link between a user and the ladder. It stores the current position in the respective ladder and any challenges they are, or have been, involved in.

	\item The fourth model is the challenges. This is used to store information on the user that was challenged, who challenged them, booking details, timing information and current status of that challenge. This is also used to track past and pending challenges for a user. They are tied to a user’s position model, allowing the service to update results after a challenge has been resolved.

	\item The final models are used for storing booking details. This data is retrieved from the \textit{Booking Facilities} microservice after a challenge has been created. The booking model stores information relevant to the new booking, such as the venue and sport. It also stores the booking ID in the case when the booking may be freed up, such as if a user concedes their match before the arranged booking time.

\end{itemize}

\par
The design, although complicated, is required for the intricacies of the requirements laid out in the specification. The design allowed for some flexibility in rankings, as the results of all matches are stored as well as positioning. This means the mechanism could be switched after a ladder has been created. A number of edge cases are handled in the challenge or ranking repository code, this keeps all logic in a single class reducing duplication and helping maintainability.

\subsubsection{User Interface}
\par
The Ladders microservice is a user facing service which users primarily interact with, so this was a primary focus. The user has three core sections to view and interact with: ladders, profiles and challenges.

\par
Each of these core views required the basic CRUD (create, read, update and delete) operations. These were scaffolded using the .NET Core Entity Framework, which helped in the rapid prototyping of the service. The challenges area required an additional view for conceding; this confirms whether the user wishes to automatically give their competitor the win. Ladders also required extra views: A view allowing users to join a ladder, another for administrators to approve users who wish to join the ladder, and an interface to remove a user.
