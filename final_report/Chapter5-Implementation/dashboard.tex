\subsubsection{MVC Architecture}
\par
The \textit{Health Dashboard} was included in the design of the system to allow users to both visualise and add to their activity data, personal goals and group challenges. To this end, the \textit{Health Dashboard} does not have a need for a persistence layer, as all of the data it processes is stored and accessed from other microservices on the system. This lead to the service being designed initially with one controller and its associated views, with model classes existing only to aid in manipulating retrieved data.

\par
Initially the service was to only have one controller: \textit{DashboardController}. Part-way through the project, however, it was noted that splitting off the administrative functionality of the \textit{Health Dashboard} to another controller could be worthwhile. Later on in the project it was decided that administrative functionality would be better placed in microservices that control the data, and so the concept was dropped. The name of the single controller was later changed to \textit{HomeController} for semantic and technical reasons associated with a repeated pattern in the URL (\textit{https://aberfitness.biz/dashboard/dashboard/\{action\}}).

\subsubsection{User Interface Style and AJAX}
\par
As the User Interface is a prime concern for this microservice, implementation of an open-source and professional template was high-priority once data processing spike-work was completed. A template was decided on with the aim to eventually implement it into every microservice. Unfortunately, some time after being implemented as part of the \textit{Health Dashboard}, the template was later changed to integrate better with Bootstrap 4 and to be simpler in DOM terms. While certainly a major advantage for the other microservices, it resulted in some time being lost to re-implementing the existing view files.

\par
In order to keep the user experience fluid, goal and activity data deletion and the simple steps recorder use AJAX on the webpage instead of using a form submission. The state of the \textit{Steps Taken By Day} graph was originally going to be refreshed using AJAX, however it was determined that it would be simpler to simply refresh the page using JavaScript, as some of the goal completion data would also need to be updated.

\subsubsection{Data Visualisation}
\par
Data visualisations were to be generated using a JavaScript library, primarily from data sourced from Health Data Repository. ChartJS\cite{chartjs} was chosen for this as it was simple to use and a member of the team had prior experience with it. This was used to create the \textit{Activity Breakdown} and \textit{Total Distance} graphs.

\subsubsection{Headless Testing and Mocking}
\par
It was originally intended that the \textit{Health Dashboard} would be primarily tested using a library such as \textit{PhantomJS} or \textit{CasperJS} in order to ensure that forms, in addition to navigation in general, were working correctly. This was impeded by the change of layout, and by extension the structure of the views, as it would not have been worthwhile to implement working tests before the change; once the change was completed, time constraints meant that adding tests wasn't feasible.

\subsubsection{Engagement}
\par
From the initial analysis of the functional requirements document, it was clear to that the \textit{Health Dashboard} would be an ideal landing page for users, given the requirement for a graph showing progress. As a result, some requirements from the Engagement section were integrated.

\par
The functional requirements specified in \textit{E-FR1} suggest that activity data rankings should be displayed based on users' activity data. It was not specified where this functionality should reside, but it seemed logical to make it a part of the \textit{Health Dashboard}. This was also the case for \textit{E-FR2}.

\subsubsection{Failing Gracefully}
\par
As \textit{Health Dashboard} is tightly coupled to the Health Data Repository microservice and loosely coupled to Challenges, FitBit Ingest and User Groups, returning the default ASP.NET exception page was deemed unacceptable: especially when considering that portions of the application would still be partially usable without Challenges or User Groups. As a result, certain cards do not generate on the Index page and all of the views show a "Health Dashboard is currently unavailable" message when Health Data Repository is down.