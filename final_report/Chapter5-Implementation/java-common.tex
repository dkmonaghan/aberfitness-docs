\subsubsection{Handling Dependencies and Static Analysis}
\par
At the start of the project the JavaEE development teams agreed to use \textit{Maven}\cite{Maven} to handle their dependencies. This allows the entire build process to be ran at the command line rather than within the IDE, which is essential for \textit{TravisCI} integration.

\par
Maven uses an XML file called \textit{POM.xml} to define the dependency names, versions and additional options. These are fetched from Maven Central and cached locally for future builds. To add a new dependency a developer simply visits the hosting website and copies the XML provided alongside each package. Additionally developers can use the \textit{Scope} option which instructs the packaging plugin the associated \textit{JAR} files are provided by the Payara runtime.

\par
\textit{IntelliJ} natively supports reading from Maven's \textit{POM.xml} file, which defines the list of dependencies and compilation steps. Maven also contains a plugin which allows developers to package the built files into a web archive (WAR). This did have some caveats however; running 
`\textit{mvn war:war}' would result packaging artefacts from the previous compilation, without compiling the current application. This could lead to confusion as new changes would no be reflected without `\textit{mvn compile war:war}'.

\par
Whilst working locally developers could still rely on IntelliJ's built in mechanism for packaging and deploying to a local Payara instance. This allowed for rapid development feedback and iteration without the overhead imposed by Maven.

\subsubsection{Creating Docker Images}
\par
By utilising the ability to run the entire build process on the command-line with Maven we were able to quickly add Travis CI integration. This would clean compile the entire application, run the full \textit{JUnit} test suite and lastly package a WAR file.

\par
Initially we used \textit{CheckStyle} to also enforce coding style conventions. However the rules were over-specified. Additionally, unlike other formatting tools there was no option to emit a patch-file which allows developers to automatically fix-up many errors such as incorrect white space or formatting.

\par
The Java teams migrated to the \textit{PMD}\cite{PMD} static analysis tool: this warns about unused fields and variables, unsafe constructs,and dangerous coding patterns.
The Maven test target was modified to run the \textit{PMD}\cite{PMD} static analysis tool before any unit tests. If any violations were detected the build would immediately stop, ensuring all modifications were inspected.

\par
The groups proceeded to add a \textit{Dockerfile} to build a docker image for later deployment. Payara provide multiple images on \textit{DockerHub}\cite{DockerHub_Payara}, which range from the full server to an embedded instance.

\par
Initially we copied the local web archive, which was built on Travis, into the full Payara deployment folder. This allowed us to use the web administration console to resolve various deployment issues without having to dig through log files. After moving changing the default source code layout, which resolved \textit{resources} not being included in the final archive, we successfully deployed GLaDOS to the full instance.

\par
However we could not get Fitbit Ingest to successfully deploy, they had already started implementing the OAuth flow which is required by both the Fitbit API and our own internal API. The internal SSL implementation had changed with JDK update 191, which prevented the \textit{CDI} layer from initialising the associated beans.

\par
As the bundled JDK version is controlled by Payara in the base image we proceeded to look for an alternative solution. We decided against modifying the image to transplant a specific JDK at build time. Instead, we found that there was an open upstream issue\cite{payara_ssl_issue} that used reflection to resolve the problem internally. As this fix was not released into the latest stable release we had to switch to running pre-release images for all Java EE applications.

\subsubsection{Migrating to Micro Image}
\par
The Payara Microprofile image is designed specifically to be used in Docker deployments. Whilst this does not implement the full EE, the memory usage is ten times lower at 90MB. The profile still provided the required set of services for the application so there was a significant benefit to completing this migration. However trying to deploy to the micro instance resulted in an exception being thrown whilst completing the implicit bean discovery phase. 

\par
The CDI 1.1 specification and above requires the server to automatically scan for injection points and pair it with matching enterprise Java beans (EJBs). This can fail with older dependencies, so we initially suspected one of the dependencies did not correctly use the \textit{beans.xml} annotation to turn this off.
\newline
Switching off bean discovery and manually annotating them allowed the ingest service to correctly deploy. However bean discovery is required for Facelets 4.0 and an exception will be thrown if it is not enabled. A new project was created with all but the essential dependencies removed, and we discovered the exception was still thrown which only added to the confusion. 

\par
By looking at the package namespaces associated with the injection points and running `mvn dependency:tree' we could see that `\textit{org.sonatype.guice}' was the culprit. Walking up the tree we discovered that the WAR plugin was being included in our packaged dependencies. At deployment the microprofile server tried to start a Maven instance and failed, thus removing this dependency correctly allowed the application to start. The development teams realised that the plugin was included in our installed copies so we could still continue to use it.

\label{JTA_Targets}
\subsubsection{Setting up JTA Targets}
Both services started off by instantiating the single JDBC connection through the driver manager as required. This had some caveats however: Firstly, the JDBC driver had to be distributed and managed within the web archive. Secondly, there was no form of connection pooling to allow for scaling at the persistence layer.

\par
Traditionally a connection pool is created using the administration web GUI. However this would require a developer set up the pool each time continuous deployment finished on a full instance. The microprofile we had just switched to did not provide a web GUI at all.

\par
Glassfish allows a web application to specify resources found on the server using `glassfish-resources.xml'. This XML file can also use system environment variables allowing us to protect and specify database credentials on the deployment targets. The examples found online primarily pertained to a \textit{MySQL} deployment, so multiple iterations of testing and deployment were required to get the pool working successfully. This switch allows an application to detect and create connection pools at deployment, allowing the images to rapidly redeploy facing another database instance.