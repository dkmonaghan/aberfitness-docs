\subsubsection{Handling Dependencies and Static Analysis}
At the start of the project, the \textit{Java EE} development teams agreed to use \textit{Maven}\cite{Maven} to handle their dependencies. This allows the entire build process to be triggered from the command line rather than within an IDE, which is essential for TravisCI integration.

Maven uses an XML file called \textit{POM.xml} to define the dependency names, versions and additional options. These are fetched from Maven Central and cached locally for future builds. To add a new dependency, a developer simply visits the hosting website and copies the XML provided alongside each package. Additionally, developers can use the \textit{Scope} option, which instructs the packaging plugin that the associated \textit{JAR} files are provided by the Payara runtime.

\textit{IntelliJ} natively supports reading from Maven's \textit{POM.xml} file. Maven also contains a plugin which allows developers to package the built files into a web archive (WAR). This did have some caveats however, running `\textit{mvn war:war}' would result packaging artefacts from the previous compilation without compiling the current application. This could lead to confusion as new changes wouldn't be reflected without running `\textit{mvn compile war:war}'.

While working locally, developers could still rely on IntelliJ's built-in mechanism for packaging and deploying to a local Payara instance. This allowed for rapid development feedback and iteration without the overhead imposed by Maven.

\subsubsection{Creating Docker Images}
By utilising the ability to run the entire build process on the command-line with Maven, we were able to quickly add TravisCI integration. This would delete previously built artefacts, compile the entire application, run the full \textit{JUnit} test suite and package a WAR file.

The development teams proceeded to add a \textit{Dockerfile} to build a Docker image for later deployment. Payara provides multiple images on \textit{DockerHub}\cite{DockerHub_Payara}, which range from the full server to an embedded instance.

Initially the local web archive which was built by Travis was copied into the full Payara deployment folder. This allowed us to use the web administration console to resolve various deployment issues without having to search through log files. After changing the default source code layout, which resolved \textit{resources} not being included in the final archive, we successfully deployed \textit{GLaDOS} to the full instance.

\textit{Fitbit Ingest}, however, could not be successfully deployed. Work had commenced on implementing the OAuth flow which was required by both the Fitbit API and our own internal API; the internal SSL implementation had changed with JDK update 191, which prevented the \textit{CDI} layer from initialising the associated beans.

As the bundled JDK version is controlled by Payara in the base image, we proceeded to look for an alternative solution. It was decided that modifying the image to transplant a specific JDK at build time would be a poor solution. Instead, an open upstream issue\cite{payara_ssl_issue} that used reflection to resolve the problem internally was discovered. As this fix was not released into the latest stable release, a change to running pre-release images for all \textit{Java EE} applications was required.

\subsubsection{Migrating to Micro Image}
The Payara Microprofile image is designed specifically to be used in Docker deployments. While this does not implement the full \textit{Java EE} specification, the memory usage is ten times lower at 90MB. The profile still provides the required set of services for the application, so there was a significant benefit to completing this migration. An issue with this was discovered when trying to deploy to the micro instance, as an exception would be thrown while completing the implicit bean discovery phase.

The CDI 1.1 specification and above requires the server to automatically scan for injection points and pair it with matching Enterprise Java Beans (\textit{EJBs}). This can fail with older dependencies, so it was initially suspected that one of the dependencies was not correctly using the \textit{beans.xml} annotation to turn this off.

Switching off bean discovery and manually annotating them allowed the ingest service to correctly deploy. Bean discovery is required for Facelets 4.0 however, and an exception will be thrown if it is not enabled. A new project was created with all but the essential dependencies removed, and it was discovered that the exception would still be thrown.

By looking at the package namespaces associated with the injection points and running `mvn dependency:tree', it was discovered `\textit{org.sonatype.guice}' was responsible for the issues. Walking through the printed dependency tree, it was discovered that the WAR plugin was being included in our packaged dependencies. At deployment, the microprofile server tried to start a Maven instance and failed; removing this dependency correctly allowed the application to start. The development teams realised that the plugin was included in the copies distributed with their operating systems, so it could continue to be used.

\label{JTA_Targets}
\subsubsection{Setting up JTA Targets}
Both services started off by instantiating the single JDBC connection through the driver manager as required. This had some caveats however. Firstly, the JDBC driver had to be distributed and managed within the web archive. Secondly, there was no form of connection pooling to allow for scaling at the persistence layer.

Traditionally a connection pool is created using the administration web GUI. This would be problematic, as it would require a developer set up the pool each time continuous deployment finished on a full instance.  It was decided not to investigate this solution further, as the Payara micro image does not provide a web GUI at all.

Glassfish allows a web application to specify resources found on the server using `\textit{glassfish-resources.xml}'. This XML file can also use system environment variables, which allowed for the protection and specification of database credentials on the deployment targets. The examples found online primarily pertained to a \textit{MySQL} deployment, so multiple iterations of testing and deployment were required to get the pool working successfully. This switch allows an application to detect and create connection pools at deployment, allowing the images to rapidly redeploy facing another database instance.
