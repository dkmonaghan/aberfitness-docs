The Layout Service provides an API to enable navigation and sidebar content to be re-used throughout the system. A management interface is provided to allow administrators to add, edit, and remove links to applications and pages within those applications. Access levels can be defined on each link defining whether or not the link should be displayed given the user's account type.

A NuGet package was developed allowing all .NET applications within the system to load and display the sidebar content in a consistent manner, whilst also introducing a number of stylistic and behavioural changes through the inclusion of assets such as CSS and JavaScript.

Due to the limited UI functionality of the applications developed in Java, and the anticipated low frequency at which users would access those applications, it was not deemed necessary to develop a package to encapsulate this functionality. Instead, similar styling was applied to the pages, but with reduced functionality. Links would not be retrieved from the Layout Service, but rather a small number of application-specific links would be shown, in addition to a link to enable users to return to the enhanced version of the user interface.

Whilst such an approach allows for the system to relatively easily ensure a consistent look and feel site-wide, whilst avoiding the necessity of re-deploying every application with a UI should the contents of the navigation sidebar need to be updated, there are also a number of disadvantages:
\begin{itemize}
    \item An additional Maven (or other Java package manager) package would need to be developed should the system be expanded to include more services written using Java.
    \item Fundamental changes to the HTML structure of the navigation sidebar require that every .NET application within the system be re-deployed after updating to the latest NuGet package. An alternative approach whereby the Layout Service would return HTML to be displayed directly in the browser, rather than just content to be inserted into an HTML template, could reduce the frequency of such re-deployments.
    \item Every page load within the system necessitates an API call to the Layout Service before the page can be rendered, which could quickly become a bottleneck should there be a large number of users accessing the system at once. A simple improvement could be made to the NuGet package, causing the response from Layout Service to be cached between page loads for a reasonable amount of time, thus vastly reducing the number of API calls made to Layout Service.
\end{itemize}
