In contrast to the experience with Java EE, the implementation process in .NET Core was much more developer friendly.  As the project progressed, services that were yet to have implementation work started were changed to be implemented in .NET Core rather than Java EE.  This allowed remaining development work to progress at a rate much faster, and to a higher quality, than if those services had been written in Java EE.

The use of EntityFramework, coupled with the Repository pattern\cite{dotnet_repository_pattern} recommended by Microsoft, made database operations simple and intuitive.  Switching to our chosen database platform, MariaDB, was made possible via the use of the \textit{Pomelo EntityFrameworkCore MySQL connector}\cite{Pomelo}, a NuGet package which allows the native entity framework to work with MySQL APIs, which MariaDB maintains compatibility with.

Development work was accelerated through the use of the .NET Core scaffolding, a code generation tool that automated the creation of Controllers and Views once a Model had been defined.  This removed the need for repetitive bootstrapping of similar classes, and provided a consistent starting point from which behaviour and presentation could be easily customised.

Creating configuration files for Travis, our Continuous Integration provider, was extremely simple - once the C# environment had been chosen, only two commands were required to install all dependencies and run the test suites.

Building Docker images was also a simple process using the example files provided by Microsoft\cite{dotnet_docker}, requiring only minor modification to expose additional ports to allow HTTPS traffic.
