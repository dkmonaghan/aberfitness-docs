\subsubsection{MVC Architecture}
\par
The \textit{Booking \& Facilities} microservice was created to allow users to make, view, modify, and remove their bookings for sport activities at their local sports venue. As bookings are created from the microservice or the \textit{Sports Ladder}, it made sense for the \textit{Booking \& Facilities} system to have its own persistence layer that stored data on bookings, venues, sports and facilities. 
\par
Thus \textit{FacilitiesController}, \textit{VenuesController}, \textit{SportsController},  and \textit{BookingsController} were created, each performing basic CRUD operations on their respective areas with a view and API for each. A simple status API was also written to give the microservice's current internal status. 
\par 
The system design went through several iterations before a design was finalised. One particular issue stemmed from uncertainty regarding how facilities and sports would link together, due to ambiguity in the client's requirements. Following clarification from the client, the model was reworked so that a facility could only be associated with a single sport. The final implementation allows multiple facilities with the same name to exist within a venue, but a facility-sport combination must be unique within that venue.
\par 
As a booking takes one specific facility, it was logical to assign a venue and a sport to a facility, and for the former two to simply contain an ID and name field. Normalising the data resulted in more complexity in the controllers, but simplified the data persisted.
\par
Another aspect of the model which was redesigned was a booking requiring an end date, alongside the pre-existing start date. This was done to accommodate the concept of a facility block; allowing an administrator to prevent booking a facility for a period of time. This would also remove any previously created bookings during the block. To implement this a boolean was added to the \textit{Booking} model specifying that the booking exists as a block. When booking normal slots the user still only provides a start date, as bookings are designed to last for exactly an hour.
\subsubsection{API Endpoints}
\par 
Several API calls to \textit{Booking \& Facilities} are made from the microservice itself specifically to get the sports, times, and update the booking's details. The only other microservice which interacts with the \textit{Booking \& Facilities} system was the \textit{Sports Ladder}. This relied on the ability to view all booking data (sports, venues, facilities and bookings/blocks), as well as create and delete bookings within the ladder. The API accommodates for this, with some slight adjustments made during the implementation of the \textit{Sports Ladder}. 
\subsubsection{User Interface Style}
\par 
Users were also required the ability directly create and modify their own bookings. The requirements also made it necessary for an administrator to delete, modify and see all user bookings. Due to the basic nature of how this data was handled there were complex visualisations. Thus all data is displayed in tabular form. New entries are created and updated with simple forms.
