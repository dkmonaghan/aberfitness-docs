\subsubsection{MVC Architecture}
\par
The \textit{Booking} microservice was created to allow users to make, view, modify, and remove their bookings for sport activities in their various local sports venues. Due to the fact that booking data was going to be created from either the booking microservice's interface or the \textit{Sports Ladder}, it made sense that the \textit{Booking} system had its own persistence layer to store data on bookings, venues, sports and facilities. 
\par
Thus \textit{FacilitiesController}, \textit{VenuesController}, \textit{SportsController},  and \textit{BookingsController} were created, each performing basic CRUD operations on their respective areas with a view for each. An API was created for each, with a simple status API to give the microservice's current online status. 
\par 
The system design went through several iterations before a design was finalised. One particular issue stemmed from uncertainty regarding how facilities and sports would link together due to ambiguity in the client's requirements. After clarification on these requirements the model was reworked so that a facility could only be associated with a single sport. The result is that within a venue there can be multiple facilities of the same name, but a facility-sport combination must be unique within a venue. 
\par 
As a booking takes one specific facility, it was logical to assign a venue and a sport to a facility, and for the former two to simply contain an ID and name field. This degree of normalisation resulted in more complex functions in the controllers, but in return added a level of simplicity and data integrity. 
\par
Another section of the model which involved a redesign was a booking requiring an end date, alongside the pre-existing start date. This was done to accommodate the concept of a facility block; this was designed to allow an admin to stop a facility from being booked for a period of time which would also remove any previously created bookings. For this purpose, a boolean was added to the Booking model specifying that the booking exists as a block. When creating a regular booking, a user still only provides a start date, as bookings are designed to last for exactly an hour. 
\subsubsection{API Endpoints}
\par 
Several API calls to \textit{Booking} are made from the microservice itself specifically to get the sports, times, and to update the bookings. The only other microservice which interacts with the \textit{Booking} system was the \textit{Sports Ladder}. This relied on the functionality to view all of the data (sports, venues, facilities and bookings/blocks), as well as create and delete bookings. The API accommodates for this, with some slight adjustments made during the implementation of the \textit{Sports Ladder}. 
\subsubsection{User Interface Style}
\par 
Beyond access via the API endpoints, functionality was required for a user to directly make and interact with their own bookings. The requirements also made it necessary for an admin to be able to delete, modify and see all user bookings. Due to the basic nature of how this data was to be handled, there was no need for any complex visualisations. Thus all data is displayed in and removed via tables and created and updated from within simple forms. 
