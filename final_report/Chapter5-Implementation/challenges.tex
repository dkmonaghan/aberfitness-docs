\subsubsection{Overall Functionality}
\par
The \textit{Challenges} microservice was created to set personal and group challenges. When creating a challenge, a member has the ability to select an activity, goal metric and goal within a time frame. The group challenges are specific to a single group, and can only be joined by users within that group. \textit{Challenges} is the only microservice to use the coordinator user type, with coordinators being able to create group challenges. 

\subsubsection{MVC Architecture}
\par
As the microservice requires CRUD operations on data, it was essential that a persistence layer was present.
Initially the microservice consisted of \textit{Challenge} and \textit{Activity} model-view-controllers, but this proved to be inadequate when it came to database normalisation. The architecture underwent many alterations, eventually resulting in \textit{UserChallenge}, \textit{ChallengeManage}, \textit{Activity} and \textit{GoalMetric} as the final model-view-controllers, with \textit{ChallengeManage} only named as such due to an issue with \textit{nginx}. The \textit{UserChallengeController} is used to keep track of a user's group and personal challenges.

\par
The index pages for \textit{UserChallenge} and \textit{ChallengeManage} display all of the user's challenge information and group challenges respectively. The \textit{ChallengeManageController} is used to handle all existing challenges. A challenge is assigned both a \textit{GoalMetricId} and an \textit{ActivityId}. The \textit{ActivityController} and \textit{GoalMetricController} allow admins and coordinators to modify activities and goal metrics. The activities are retrieved from the \textit{Health Data Repository} and the goal metrics must be manually entered to match the database key.

\subsubsection{API Access}
\par
The microservice requires access to the \textit{Health Data Repository} to retrieve data in order to update the progress of a challenge. When the user challenges page or API endpoint is accessed, their challenges are updated with the most recent data.
Access is also required to \textit{User Groups} to retrieve information about a user or group. A user will only see challenges for their group on the group challenges page. 
In addition access is required to the \textit{Communications} microservice, in order to send notifications to users when their challenges close.

\subsubsection{API Endpoints}
\par
The \textit{Health Dashboard} requires access to the \textit{Challenges} microservice to display personal and group challenges and create the former. Displaying challenges on the \textit{Health Dashboard} required retrieving a user's personal and group challenges, which necessitated \textit{GET} endpoints. \textit{POST} endpoints were needed to create a challenge and associated user challenge. As the challenge can only be created for specific activities, the \textit{Health Dashboard} must also retrieve a list of available activities from a \textit{GET} endpoint.
Although the functionality was ultimately not implemented, the \textit{PUT} endpoint was created to allow the \textit{Health Dashboard} microservice to update existing personal challenges.
A status endpoint was necessary for \textit{GLaDOS}, so that the current status of the microservice could be displayed.

\subsubsection{User Interface}
\par
As this microservice is responsible for managing user data, the majority of the user interface was generated using scaffolding functionality. Most of the pages were adapted to display the desired contents, depending on the identity of a user.
