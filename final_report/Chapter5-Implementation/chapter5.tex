\chapter{Implementation}

\section{Deployment}

\section{Java EE}
    \subsection{Common Components}
        \subsubsection{Handling Dependencies and Static Analysis}
        \par
        At the start of the project the JavaEE development teams agreed to use \textit{Maven}\cite{Maven} to handle their dependencies. This allows the entire build process to be ran at the commandline rather than within the IDE, which is essential for \textit{TravisCI} integration.

        \par
        Maven uses an XML file called \textit{POM.xml} to define the dependency names, versions and additional options. These are fetched from Maven Central and cached locally for future builds. To add a new dependency a developer simply visits the hosting website and copies the XML provided alongside each package. Additionally developers can use the \textit{Scope} option which instructs the packaging plugin the associated \textit{JAR} files are provided by the Payara runtime.
        
        \par
        \textit{IntelliJ} natively supports reading from Maven's \textit{POM.xml} file, which defines the list of dependencies and compilation steps. Maven also contains a plugin which allows developers to package the built files into a web archive (WAR). This did have some caveats however; running 
        `\textit{mvn war:war}' would result packaging artefacts from the previous compilation, without compiling the current application. This could lead to confusion as new changes would no be reflected without `\textit{mvn compile war:war}'.

        \par
        Whilst working locally developers could still rely on IntelliJ's built in mechanism for packaging and deploying to a local Payara instance. This allowed for rapid development feedback and iteration without the overhead imposed by Maven.

        \subsubsection{Creating Docker Images}
        \par
        By utilising the ability to run the entire build process on the command-line with Maven we were able to quickly add Travis CI integration. This would clean compile the entire application, run the full \textit{JUnit} test suite and lastly package a WAR file.

        \par
        Initially we used \textit{CheckStyle} to also enforce coding style conventions. However the rules were over-specified. Additionally, unlike other formatting tools there was no option to emit a patch-file which allows developers to automatically fix-up many errors such as incorrect whitespace or formatting.

        \par
        The Java teams migrated to the \textit{PMD}\cite{PMD} static analysis tool: this warns about unused fields and variables, unsafe constructs,and dangerous coding patterns.
        The Maven test target was modified to run the \textit{PMD}\cite{PMD} static analysis tool before any unit tests. If any violations were detected the build would immediately stop, ensuring all modifications were inspected.

        \par
        The groups proceeded to add a \textit{Dockerfile} to build a docker image for later deployment. Payara provide multiple images on \textit{DockerHub}\cite{DockerHub_Payara}, which range from the full server to an embedded instance.

        \par
        Initially we copied the local web archive, which was built on Travis, into the full Payara deployment folder. This allowed us to use the web administration console to resolve various deployment issues without having to dig through log files. After moving changing the default source code layout, which resolved \textit{resources} not being included in the final archive, we successfully deployed GLaDOS to the full instance.

        \par
        However we could not get Fitbit Ingest to successfully deploy, they had already started implementing the OAuth flow which is required by both the Fitbit API and our own internal API. The internal SSL implementation had changed with JDK update 191, which prevented the \textit{CDI} layer from initalising the associated beans.

        \par
        As the bundled JDK version is controlled by Payara in the base image we proceeded to look for an alternative solution. We decided against modify the image to transplant a specific JDK at build time. Instead we that upstream had opened an issue \cite{payara_ssl_issue} and used reflection to resolve the problem internally. As this fix was not released into the latest stable we had to switch to running pre-release images for all JavaEE applications.

        \subsubsection{Migrating to Micro Image}
        \par
        The Payara Microprofile image is designed specifically to be used in Docker deployments. Whilst this does not implement the full EE, the memory usage is ten times lower at 90MB. The profile still provided the required set of services for the application so there was a significant benefit to completing this migration. However trying to deploy to the micro instance resulted in an exception being thrown whilst completing the implicit bean discovery phase. 

        \par
        The CDI 1.1 specification and above requires the server to automatically scan for injection points and pair it with matching enterprise Java beans (EJBs). This can fail with older dependencies, so we initially suspected one of the dependencies did not correctly use the \textit{beans.xml} annotation to turn this off.
        \newline
        Switching off bean discovery and manually annotating them allowed the ingest service to correctly deploy. However bean discovery is required for Facelets 4.0 and an exception will be thrown if it is not enabled. Adding to the confusion we created a new project and removed all but the essential dependencies to discover the exception was still thrown.
        
        \par
        By looking at the package namespaces associated with the injection points and running `mvn dependency:tree' we could see that `\textit{org.sonatype.guice}' was the culprit. Walking up the tree we discovered that the WAR plugin was being included in our packaged dependencies. At deployment the microprofile server tried to start a Maven instance and failed, thus removing this dependency correctly allowed the application to start. The development teams realised that the plugin was included in our installed copies so we could still continue to use it.

        \label{JTA_Targets}
        \subsubsection{Setting up JTA Targets}
        Both services started off by instantiating the single JDBC connection through the driver manager as required. This had some caveats however: Firstly, the JDBC driver had to be distributed and managed within the web archive. Secondly, there was no form of connection pooling to allow for scaling at the persistence layer.

        \par
        Traditionally a connection pool is created using the administration web GUI. However this would require a developer setup the pool each time continuous deployment finished on a full instance. The microprofile we had just switched to did not provide a web GUI at all.

        \par
        Glassfish allows a web application to specify resources found on the server using `glassfish-resources.xml'. This XML file can also use system environment variables allowing us to protect and specify database credentials on the deployment targets. The examples found online primarily pertained to a \textit{MySQL} deployment, so multiple iterations of testing and deployment were required to get the pool working successfully. This switch allows an application to detect and create connection pools at deployment, allowing the images to rapidly redeploy facing another database instance.

    \subsection{Fitbit Ingest Service}
        \subsubsection{To Do}

    \subsection{GLaDOS}
        \subsubsection{REST API}
        \par
        The primary role of GLaDOS is to store audit data so that users can view how their data was viewed and modified. Whilst there are native implementations, such as the Java Messaging Service, these require the calls be proxied into a local JVM instance to handle the transport component. We chose to serialise into JSON and use REST as the transport mechanism as all applications had native support for this setup.

        \par
        \textit{JAX-RS} is an API specification for JSON parsing in Java, additionally Payara provides \textit{Jersey} as the implementation of this API at runtime. This avoided us having to package additional dependencies into the web archive. This also allows us to build standard response pages based on the error code internally generated, avoiding having to develop error views.        

        \subsubsection{Unit Testing and Mocks}
        \par
        A new class abstracting the database layer allowed us to avoid writing entity handling at this. We instead focused on writing unit tests which verified the various endpoints correctly serialised or de-serialised data. \textit{Mockito}\cite{Mockito} allows developers to inject mock objects by specifying the class which the test fixture is using.

        \par
        We could not use dependency injection within the endpoint implementation, as the instantiation point is internal to the framework. This doesn't prevent us from using mocks as Mockito allows a developer to inject them through the reflection methods built into the library. A database call such as `\textit{db.getEntry(id)}' could be tested to see which status codes would be sent and verify the data,if any, was serialised correctly.

        \subsubsection{Integration Testing with Arquillian}
        \textit{Arquillian}\cite{Arquillian} allows developers to specify the classes to archive and deploy within the text fixture. Payara also provides an embedded Arquillian test container which can be ran at test time to exercise sections of a full system. This would allow us to write integration tests for an end-to-end operations, such as POSTing data to an endpoint and ensuring it is written to the persistence layer.

        \par
        There was extensive documentation on the methods required to pack an archive, but limited examples on handling Maven dependencies. As the embedded instance provides no runtime methods we were having problems getting the container to correctly run. Developers could specify the full class paths for their dependencies to ensure the were packaged, but this was extremely verbose and fragile.

        \par
        We looked into using the Maven dependency resolver, which allows developers to get a list of runtime Jars and package them into the archive. However this would led to another set of problems where Maven was trying to export them into a \textit{.zip} format, then throw an exception because this was not supported.

        \par
        With no easy was to control the format that runtime dependencies were exported in, combined with a lack of documentation pertaining specifically to Maven we agreed to abandon this method of automated testing. This would give us more time for implementing other components within the deadline. We opted to rely on manual integration testing for the duration of the project instead.

        \subsubsection{Entity Management}
        This service only persists Audit Data, therefore we ultimately need to persist a single entity. Whilst many ORM solutions exists for Java,such as \textit{Hibernate}, we decided against them due to the implementation overhead that was required.

        \par
        Instead we relied on the persistence layer provided in the EE specification. Using the connection pools which were setup for both Java services (see \ref{JTA_Targets}), we could specify which JTA the injected entity manager should use.

        \par
        The table schema is managed through annotations on an entity class. This also allows the implementation provided by Payara, \textit{EclipseLink}, to create the tables required at deployment time. As the framework relies on the field types to determine which underlying storage to use there is a hidden pitfall. If the type does not have a native serialisation Java will try to persist this using binary data.

        \par
        Instants are used to specify timestamps on logs, these are cross compatible as they use \textit{ISO 8601}\cite{ISO_8601}, a string specifier for absolute time points. This format allows both .Net and Java application to send timestamps in the following format \textit{2018-11-28T12:04:14Z}. However as this type was added in JDK 8 there is no native storage conversion built into the entity framework.

        \par
        Two adaptors were written based on the \textit{AttributeConverter} interface. These provide methods for marshalling and un-marshalling Instants into String objects, which the entity manager can easily store. Annotating the field installed the converting class and correctly updated the schema. This also has the added benefit of making the stored timestamps human readable within the database, which extremely helpful for debugging.

        \subsubsection{Developing Facelets}
        \par
        GLaDOS also provides a page that allows a user to retrieve audit data associated with their account. Administrators can lookup any users audit data by user their unique ID too. In addition there is a status page which allows anyone to view the status of all other micro-services without logging in.

        \par
        The front-end of GLaDOS use facelets to implement the MVC pattern. A backing bean for the user data uses named queries from the persistence layer to retrieve data into a list of entity objects. We switched to using \textit{PrimeFaces}\cite{Primefaces}, which is a fork of the now deprecated \textit{RichFaces}. This allows us to use components such as \textit{DataTables} that are implicitly reactive rather than us developing our own supporting CSS.

        \par
        The Fitbit Ingest Service had completed OAuth implementation for connecting to internal and external APIs. This was ported across to GLaDOS and further modified. As the ingest service has no front-end they don't need to persist \textit{JSON Web Token (JWT)}. Additional methods were written specifically for this microservice. These include using the HTTP session handling built into the web framework to persist the token on the server. Another helper class was written which validates the stored JWT as the user navigates through protected pages, or POSTs any requests.

        \par
        Service statuses were implemented using a singleton Java bean which is instantiated at deployment. Using the scheduling capability provided for EJBs we poll all services every 20 seconds in a seperate thread. All services implement an endpoint at \textit{/api/status} that returns a 200 or 204. This is stored until the page is retrieved where the backing bean, acting as a presenter forward the results on.

        \par
        As the application stack uses an Nginx instance to reverse proxy queries the URLs must take this into account when they are generated within the facelet. .Net makes this trivial by calling `\textit{App.UsePathBase(URL)}' at startup, however Java applications rely on the context root. This was interfering with the URL rewriting that Nginx performs resulting in the service using the 
        `\textit{\texttt{\textbackslash}glados\texttt{\textbackslash}glados\texttt{\textbackslash}}' base path.

        \par
        Initially we used an environment variable to manually generate links between the pages of the service and set the context root to 
        `\textit{\texttt{\textbackslash}}'. However, this quickly proved to be untenable when using forms, as POST requests were being sent to 
        `\textit{\texttt{\textbackslash}destination}' 
        rather than 
        `\textit{\texttt{\textbackslash}glados\texttt{\textbackslash}destination}'. Ultimately to work around this an exception was added to Nginx configuration to avoid re-writing any URLs for GLaDOS, and instead rely on the service to correctly address all requests. This allowed us to switch back to generating addresses based on the \textit{outcome} tag and use forms on the service.



\section{.NET Core}
    \subsection{}{Common Components}
some blurb about .net core here

    \subsection{Booking Facilities}

    \subsection{Challenges}

    \subsection{Communications}

    \subsection{Gatekeeper}

    \subsection{Health Data Repository}

    \subsection{Health Dashboard}

    \subsection{Ladders}

    \subsection{User Groups}
